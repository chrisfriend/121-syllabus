\documentclass[12pt,oneside]{amsart}	%defines this as an article
\usepackage{chrisfriend-comp} %provides formatting declarations for page, headers, figures, textcolor, comments, and bibliographic styles
\usepackage{chrisfriend-OTF-support} %provides support for OTF system fonts; incompatible with latex, rtf2latex, & ht4latex
%\usepackage[utf8]{inputenc} %support for smallamp?

%\usepackage{tabularx}
\usepackage{tabulary} % allows for the tables I make rubrics with
%\usepackage{supertabular}
\usepackage{xtab} % allows tables to span pages
\usepackage{booktabs} % allows fancy lines in tables
%\usepackage{rotating} % allows landscape tables
\usepackage{lscape} % allows rotated longtables
\usepackage{multirow} % allows rowspanning
\usepackage{enumitem} % helps with the overview
%\usepackage{paralist}
\usepackage[nodate]{datetime} % allows the \currenttime command; nodate tells it not to mess up date settings
     \usepackage{pdflscape} % turns landscape pages sideways in PDFs; avoids kinks in neck.
\usepackage{tagging} % Allows conditional text inclusion; see http://www.ctan.org/tex-archive/macros/latex/contrib/tagging
\usepackage{draftwatermark}

\usepackage{acronym}
\acrodef{slu}[\textsc{slu}]{Saint Leo University}
\acrodef{phrg}[\textsc{phrg}]{Prentice Hall Reference Guide}
\acrodef{awr}[\textsc{awr}]{The Academic Writing Reader}
\acrodef{121}[\textsc{eng~121}]{Academic Writing}
\acrodef{rr}[\textsc{rr}]{Reading Response}
\acrodef{lrc}[\textsc{lrc}]{Learning Resource Center}


%%%%%%%%%% Adjust for multiple classes
% 
% Use this line to specify section number:

\usetag{ca14} % Fall 2014 is ca09, ca10, and ca14 for eng 121

% Use structure inside this comment environment to prepare multiple versions of text:
%
% \begin{taggedblock}{ca09}
%	content here for section ca09 only
% \end{taggedblock}
%
% or for one-liners:
%
% \tagged{ca09}{text for ca09 only}
%
%%%%%%%%%%%%% End multiple Versions content

\title[\textsc{eng}~121 Syllabus]{Course Syllabus: Academic Writing I}
\chead{\scriptsize{\fontspec[Numbers=Lining]{ITC Berkeley Oldstyle Std}\MakeUppercase{ENG~121 Syllabus} }}%(Rev. \number\day\space\monthname\space\number\year)}}

  
\begin{document}
%\bibliographystyle{abbrv}

\vspace{-2in}
\begin{center}
\huge
\includegraphics[height=2\baselineskip]{leo-logo.pdf}

\textbf{Course Syllabus: Academic Writing I}
\end{center}
 

\label{sec:about_the_course}
\vspace{1.5\baselineskip}
\begin{center}
\begin{minipage}{0.75\textwidth}
%	|\hfill|
	\begin{description}[align=right, labelwidth=*, labelindent=0.9in, leftmargin=1in]
	\item [Course Section] 	\tagged{ca09}{\textsc{eng 121.ca09}}
		\tagged{ca10}{\textsc{eng 121.ca10}}
		\tagged{ca14}{\textsc{eng 121.ca14}}
	\item[Meeting]
		\tagged{ca09}{\textsc{tr} 11:00--12:20, Daniel A.\ Cannon Memorial Library, Southard room}
		\tagged{ca10}{\textsc{mwf} 11:30--12:20, Daniel A.\ Cannon Memorial Library, Southard room}
		\tagged{ca14}{\textsc{mwf} 12:30--13:20, school of business building, room 225}
	\item[Prerequisite] Passing grade in \textsc{eng 002} or satisfactory score on the English Placement Test
	\item[Term] Fall 2014
	\item [Credit Hours] 3
	\vspace{.5\baselineskip}
	\item[Professor] Chris Friend
	\item[Email] \href{mailto:christopher.friend@saintleo.edu}{christopher.friend@saintleo.edu}
	\item[Office] Saint Edward Hall 238
	\item[Office Hours] \textsc{mf} 14:00--15:00 and \textsc{tr} 13:00--15:00; appointments strongly recommended. Visit \href{http://friend.lattiss.com}{http://friend.lattiss.com} for availability.
\end{description}
\end{minipage}
\end{center}
\vspace{0.75\baselineskip}
\thispagestyle{empty}

%\section{Overview} % (fold)
\section{Course Description} % (fold)
\label{sub:course_description}
Academic Writing I is designed to teach students to communicate effectively in an academic environment. The goal of the course is to provide instruction, practice, and discussion to improve students' communication skills. Students will write for a variety of purposes and audiences and in a variety of rhetorical modes. The focus of the course is on practical, relevant, academic writing skills. Although good prose models are used throughout the course, the students' writing is the primary focus. All students will present one formal speech. 

See Table~\ref{tab:overview} for an overview of the topics studied in this course and a list of the major required papers.

% subsubsection my_version (end)
% subsection course_description (end)


% section about_the_course (end)
\section{Student Learning Outcomes}\label{outcomes}
Through successful completion of this course and its activities, you should be able to
\begin{itemize}
	\item Perfect the ability to write clear theses. 
	\item Demonstrate proficiency in writing well-constructed introductory, body, and concluding paragraphs.
	\item Demonstrate the ability to revise and proofread various kinds of writing.
	\item Demonstrate proficiency in producing, revising, and editing drafts of an essay.
	\item Demonstrate proficiency in spelling, punctuation, and grammar.
	\item Improve your critical thinking, problem solving, writing style, and speaking skills.
	\item Develop proficiency in writing a five-paragraph essay in the various rhetorical modes.
\end{itemize}

\section{Key Core Values} % (fold)
\label{sec:key_core_values}
Although all six of \ac{slu}'s core values should be evident in the daily operation of our class and in every assignment you complete, the School of Arts \& Sciences has chosen two as the key core values for this course.
\begin{description}
	\item [Integrity] The \ac{slu} commitment to excellence ``demands that its members live its mission and deliver on its promise. The faculty, staff, and students pledge to be honest, just, and consistent in word and deed.'' We will demonstrate integrity by presenting our own work genuinely and our ideas honestly, both in discussion and in writing.
	\item [Respect] At \ac{slu}, ``we value all individuals' unique talents, respect their dignity, and strive to foster their commitment to excellence in our work. Our community's strength depends on the unity and diversity of our people; on the free exchange of ideas; and on learning, living, and working harmoniously.'' We will demonstrate respect in our dealings with others, including our peers with us in class and the authors whose work inspires or informs our discussion and writing.
\end{description}
% section key_core_values (end)

%\clearpage
\section{Materials for Class}
\begin{itemize}
	\item Required
		\begin{enumerate}
		\item \citeauthor{harris:aa}, \citetitle{harris:aa}, Ninth Edition (\textsc{isbn} 978-0-321-92131-4)
		\item Saint Leo University, \citetitle{saint-leo-university:2006aa} (\textsc{isbn} 0-536-97592-2)
		\item Reliable connection to the Internet outside of class. Make a plan for what you will do/use if your device or connection dies.
		\item Automated, reliable backup system. Every semester, I have a student who loses everything due to a hard drive failure. Don't be that student.
		\item Regular access to your \href{http://outlook.com/saintleo.edu}{student email account}. I check my email multiple times per day and will almost always reply within one business day. You should to check yours \emph{at least} once per day, but definitely before each class meeting. (Why not \href{http://www.saintleo.edu/media/216502/directions_for_setting_up_student_email_access.pdf}{set it up on your phone}?)
	\end{enumerate}
	\item Recommended
	\begin{enumerate}
		\item A Google account associate with your \ac{slu} email address. We will use this account for collaborative writing and to make document submission simpler. We will set this up on the second day of class.
		%\item Dropbox account into which you store all your work. This takes care of backups.
		\item Your own computer running a full (non-mobile) operating system. Some of the work we do is much simpler with new software and the ability to run multiple programs simultaneously. Phones are too limited, and tablets can get frustrating. (Campus computer labs can work in a pinch.)
	\end{enumerate}
\end{itemize}

\section{Grading \& Assessment}
%Your grade in this course will be based on two holistic grades listed in Table~\ref{tab:assignments}. Think of these like grades for a semester-long project: the components work together to build the overall value of the whole, which will be graded in this course. You will get consistent feedback throughout the semester to help ensure you are on-track for a successful grade. Additionally, each major assignment will have a specific assessment rubric, and every smaller assignment will have detailed completion guidelines that will be provided  on \href{https://webcourses2c.instructure.com/courses/985581}{Webcourses}.  The smaller assignments are designed to help you build skills and confidence as you work toward your final portfolio. They should not be dismissed.

Please note the following distinctive characteristics about grading in this course:
\begin{itemize}
	\item You can earn a D for an assignment or major component, but you cannot earn a D for this course. To pass, you must earn at least a C, or 74 points. \textbf{Earning a C$-$ is not sufficient.}
	\item The grade of NC (no credit) can be assigned at the instructor's discretion only if you complete all course work on time, participate fully, and fail to produce satisfactory work for the class.
\end{itemize}


\begin{figure}[t]
	\centering
	\subtable[Grade Calculations]{\small
		\noindent\begin{tabulary}{0.25\textwidth}{lr}
		\toprule
		\textbf{\textsc{Grade}} & \textbf{\textsc{Min.\ Points}}\\
		\midrule
			A	&	94	\\
			A−	&	90	\\
			B+	&	88	\\
			B	&	84	\\
			B−	&	80	\\
			C+	&	78	\\
			C	&	74	\\
\addlinespace
%\midrule
			C−	&	70	\\
			D+	&	68	\\
			D	&	60	\\
%			NC	&	Unsatisfactory	\\
			F	&	<60	\\
		\bottomrule
		\end{tabulary}\label{tab:final-grades}
	} %subtable
	\quad % Middle gap spacing
	\subtable[Grade Distribution]{\small
		\noindent\begin{tabulary}{\textwidth}{Lr}
			\toprule\textbf{\textsc{Component}} & \textbf{\textsc{Points}}\\
				\midrule	Products (essays \& other writing)	&	50	\\
%				\midrule	Academic Research Report	&	25	\\
%				\midrule	Genre Product \& Presentation	&	25	\\
					Process (participation \& collaboration)	&	50	\\
				\midrule	\textbf{\textsc{Total}}	& \textbf{\textsc{100}}\\
			\bottomrule
		\end{tabulary}\label{tab:assignments}
	} %subtable
	\caption{Course Grading System}
\end{figure}

\subsection{Grading Standards} % (fold)
\label{sub:grading_standards}
Participation in all activities, and successful completion of all assignments (as defined by each assignment's assessment rubric) will earn you a passing grade of C, indicating that you have achieved the expected outcomes of the course. If you do not take part in all assignments and activities, you should not expect a passing grade in the course. If the quality of your work or your participation falls below acceptable standards (i.e. if you are heading for failure), I will be sure to let you know. Along more optimistic lines, grades of B or A are used for work that is good and excellent, respectively, surpassing the basic expectations. Assignment sheets will suggest ways to exceed those expectations, so you won't have to guess. If your performance exceeds basic standards, I will be sure to let you know.

%Besides participation, all grades for this course come from products turned in at the end of the semester (see Table~\ref{tab:assignments}). This is by design, to allow you the chance to experiment and take risks as you progress through the course. If your research leads you down a dead-end, your grade will not be affected. In lieu of grades, you will receive regular feedback from your peers and instructor about what is and isn't working well with your research and various written products. Take the feedback seriously, and revise your work regularly throughout the semester so that your final portfolio best reflects your ability.

% subsection grading_standards (end)



\subsection{Expectations} % (fold)
\label{sub:expectations}

While enrolled in this course, you can expect these things from me:\footnote{The structure and approach of the Expectations section is adapted from the \href{http://www.ceball.com/classes/239/spring09/?page_id=8}{English 239 syllabus} of Cheryl E. Ball, \textsc{isu}.}
\begin{itemize}
	\item enthusiasm for learning, teaching, and writing;
	\item clarity and thoroughness in assignments, goals, and expectations;
	\item personal interest in your learning and work;
	\item flexibility, allowing you the freedom to be creative with the products you create for this course;
	\item critical feedback to help you improve your thinking and writing; and
	\item preparation to ensure a beneficial and productive semester.
\end{itemize}
If at any point you feel I am failing to meet any of those expectations, please let me know. Your feedback is the best way I can learn how to improve my teaching.

As we progress through the semester, your peers and I will expect these things from you:
\begin{itemize}
	\item consistent and active participation in class activities, including peer review assignments;
	\item informed contributions, based on sufficient preparation and consideration (i.e. doing the readings and research)
	\item an open mind, tolerant and curious about differences of opinion; and
	\item honest and polite commentary and feedback that helps your peers improve their work.
\end{itemize}

During class discussions and as you work on your assignments, keep in mind that I value these things in my students:
\begin{itemize}
	\item thought-out and supported opinions;
	\item willingness to take risks and try new approaches to solving problems, as risks often create the greatest opportunities;
	\item creativity in how you respond to the challenges created and faced by this course; and
	\item excellence in your work, showing the best you can produce.
\end{itemize}

% subsection expectations (end)



\section{Course Contents} % (fold)
\label{sec:course_contents}

The first day of class will involve discussion about what students think would be the best way for them to achieve the \nameref{outcomes}. In general, this class will consist of writing and discussing your ideas, reading the ideas of others, and then writing again to see whether or how the ideas of others integrate with your initial thinking. Because we will determine how class progresses together, and because we are interested in a diverse range of ideas, your active participation in class discussions is the most essential component of a successful semester. This importance is reflected in the grading system used for this course.

The units presented below (and the information in Table~\ref{tab:overview}) are suggestions, presented in a suggested order. We will discuss, debate, and decide how the class will actually flow on the first day of class.

\begin{table}[b]
\caption{Proposed Assignment Overview}\label{tab:overview}
	\begin{tabulary}{\textwidth}{CLL}
		\toprule	\textbf{\textsc{Weeks}}	&	\textbf{\textsc{Unit}}		&	%	\textbf{\textsc{Readings}}	&	\textbf{\textsc{Minor Assignments}}	&	
		\textbf{\textsc{Major Papers}}	\\


\midrule	1	&	Planning the Term		&	Course Calendar	\\
\midrule	2--5	&	The Rules of Writing		&	Narrative Essay	\\
					&								&	Assignment Sheets \& Rubrics \\
\midrule	6--8	&	Teaching Writing		&	Expository Essay	\\
					&							&	Document, type \textsc{tbd}	\\
\midrule	9--11	&	Writing in Society		&	Descriptive Essay	\\
					&							&	Project Proposal \\
\midrule	12--15	&	Writing for Change		&	Persuasive Essay	\\
					&							&	Campaign Plan	\\
					&							&	Oral Presentation/``Pitch'' \\
	\bottomrule
	\end{tabulary}
\end{table}


\subsection{Rules, Regulations, and Following Orders} % (fold)
\label{sub:doing_what_we_re_told}
\begin{description}
	\item[Guiding question] What should we do in this class?
	\item [Working unit] Whole class
	\item [Reading responses] Selections from these options:
	\begin{description}
		\item[Stanley Milgram] ``The Perils of Obedience'' (p.~653)
		\item[Anthony Burgess] from \emph{A Clockwork Orange} (p.~246)
		\item[Joseph M.\ Williams] ``The Phenomenology of Error'' (get from LearningStudio)
		\item[John Warner] ``Rethinking My Cell Phone/Computer Policy'' (get from LearningStudio) % https://www.insidehighered.com/blogs/just-visiting/rethinking-my-cell-phonecomputer-policy
		\item[Michael Kleine] ``What Is It We Do…?'' (get from LearningStudio)
		\item[Your choice] Find related content online
	\end{description}
	\item [Survey] What kinds of writing are done in other disciplines/classes? (Ask other teachers.)
	\item [Product] Detailed assignment plans—What will you do for each unit, and what is the focus of each? Are they built around required assignments, interesting topics, traditional writing concepts, or something else?
	\item [Essay] Narrative form telling how the process [reflected/challenged/reinforced/etc] your core values. Audience: \ac{slu} administration, other writing teachers, or someone else?
\end{description}
% subsection doing_what_we_re_told (end)

\subsection{The Value of Education} % (fold)
\label{sub:the_value_of_education}
\begin{description}
	\item[Guiding question] Is teaching writing actually important?
	\item [Working unit] Large teams
	\item [Reading responses] Selections from these options:
	\begin{description}
		\item[Jonathan Kozol] ``The Human Cost of an Illiterate Society'' (p.~158)
		\item[Temple Grandin] ``Thinking in Pictures'' (p.~208)
		\item[David Rothenberg] ``How the Web Destroys…'' (p.~182)
		\item[Your choice] Anything else in Chapter~3
		\item[Josh Keller] ``Studies Explore…'' (get from LearningStudio)
		\item[Helen Keller] ``The Day Language Came into My Life'' (p.~206)
		\item[Aldous Huxley] ``Propaganda Under a Dictatorship'' (p.~235)
		\item[Your choice] Anything in Chapter~10—The Artistic Impulse
		\item[Your choice] Find related content online
	\end{description}
	\item [Survey] Why do we teach writing? (Ask college graduates, especially those in your intended field.)
	\item [Product] Class-chosen genre; audience probably past selves, younger students, or former teachers.
	\item [Essay] Expository form defining \emph{education} and explaining its function from multiple sides, examining multiple values/perspectives.
\end{description}
% subsection the_value_of_education (end)

\subsection{Issues in Popular Culture} % (fold)
\label{sub:issues_in_popular_culture}
\begin{description}
	\item[Guiding question] How do we change society?
	\item [Working unit] Small groups
	\item [Reading responses] Selections from these options:
	\begin{description}
		\item[Juliet B.\ Schor] ``The Culture of Consumerism'' (p.~256)
		\item[Philip Slater] ``Want-Creation Fuels Americans' Addictiveness'' (p.~264)
		\item[Your choice] Anything else in Chapter~5
		\item[Your choice] Anything in Chapter~9—The Impact of Technology
		\item[Nicholas Carr] ``Is Google Making Us Stupid?'' (get from LearningStudio)
		\item[Your choice] Find related content online
	\end{description}
	\item [Survey] Your group will design its own question(s) and choose a relevant audience.
	\item [Product] Project proposal for a group-selected change agent. How do changes happen in a large [institutional/governmental/regional] scale?
	\item[Presentation] Your group will create a \href{http://www.pechakucha.org/faq}{Pecha Kucha} to ``pitch'' your project to your peers.
	\item [Essay] Descriptive form focusing on the problem. [To consider in discussion: Should the essay include the solution?]
\end{description}
% subsection issues_in_popular_culture (end)

\subsection{Personal Project Unit} % (fold)
\label{sub:personal_project_unit}
For this unit, you will create your own products, rather than contributing to group products. Whether you still work in groups as you progress will be determined by class discussion. Because the product is individual, you will have more choice regarding what you read and create.

The expectations below apply to whichever option you choose to study.
\begin{description}
	\item [Working unit] Individual
	\item [Interview] Chat with a small number of people (1--3) about the issue you chose.
	\item [Product] Plan a campaign to bring about awareness or change based on the issues studied; present that plan to the class
	\item[Presentation] Your group will create a \href{http://www.pechakucha.org/faq}{Pecha Kucha} to ``pitch'' your project to your peers.
	\item [Essay] Persuasive form, perhaps intended for \emph{The Lions' Pride} campus newspaper or an online blog posting.
\end{description}

\subsubsection{Matters of Ethics, Philosophy, and Religion (Option 1)} % (fold)
\label{sub:matters_of_ethics_philosophy_and_religion}
\begin{description}
	\item[Guiding question] How do we make complex decisions?
	\item [Reading responses] Selections from these options:
	\begin{description}
		\item[Philip Wheelwright] ``The Meaning of Ethics'' (p.~628)
		\item[Your choice] Anything else in Chapter~11
		\item[Your choice] Find related content online
	\end{description}
\end{description}
% subsubsection matters_of_ethics_philosophy_and_religion (end)

\subsubsection{The Pursuit of Justice (Option 2)} % (fold)
\label{sub:the_pursuit_of_justice}
\begin{description}
	\item[Guiding question] How do we determine what is ``fair''?
	\item [Working unit] Individual
	\item [Reading responses] Selections from these options:
	\begin{description}
		\item[Barbara Ehrenreich] ``Nickel-and-Dimed'' (p.~474)
		\item[Your choice] Anything else in Chapter~8
		\item[Coward, Ashe, or Kantowitz] from Pop Culture chapter (see top of p.~viii)
		\item[Your choice] Find related content online
	\end{description}
\end{description}
% subsection the_pursuit_of_justice (end)
% subsection personal_project_unit (end)


% section course_contents (end)

 

\section{Policies \& Miscellanea}
%\nocite{Curtis:2009uq,Tripp:2009kx,Wardle:2010fk} %ensures the syllabi I stole from will be in the Works Consulted list

\subsection{Participation}
Your attendance is mandatory, and your success in this course depends on your active engagement.  If you are absent more than three times, your final grade will be reduced by one letter grade per additional day missed; therefore, after three absences, I recommend that you drop the class. If you are absent more than five times, you risk failing the course.  If you must be absent, it is \emph{your} responsibility to complete the day's activities and contact your peers to determine what you missed and how you need to recover. Any absence will cause you to forfeit credit for any participation or activities for those days.

Absences due to university-sponsored events---such as music performances, athletic competitions, debates, and some conferences---can excuse you from certain minor assignments (but not major papers). When participating in school-sponsored events, get the appropriate form from the organization sponsor and submit it to your instructor before you miss class. Absences due to religious holidays not observed by the university should be discussed with the instructor during the first week of the semester.

\newpage
Please note these details: 
\begin{enumerate}
	\item Major assignments will be submitted online, so attendance (or lack thereof) does not affect your ability to submit work. You are still expected to turn in your work regardless of whether you are in class that day.
	\item For the purposes of this attendance policy, arriving tardy to class twice equals one absence.
	\item I do not distinguish between ``excused'' and ``unexcused'' absences. If you are not in class, we cannot benefit from your participation, and you are absent. I consider university-sponsored events (mentioned in the paragraph above) the equivalent of attendance.
\end{enumerate}

Treat participation in class activities (including discussions, peer review assignments, etc.) as evidence of attending to the course. I expect complete participation on all assignments from each student. We all know that the most boring classes are the ones where the instructor does all the talking. Don't let that become the case here. Share your thinking, provide your opinion, and join in the work. When in doubt, speak your mind---it's the only way your peers and your instructor will know what you're thinking, and the only way we can compliment, complement, or correct, as appropriate.

\subsection{Late and Make-Up Work} % (fold)
\label{sub:late_and_make_up_work}
Major writing assignments will be submitted online, and computers are good at treating deadlines as absolutes. You will not be able to submit work late; I expect that you will be prepared. Minor activities done in class are designed to take advantage of the live interactions of all students and cannot be meaningfully ``made up'' after the class has ended; therefore, there is no make-up work in this class.
% subsection late_and_make_up_work (end)

\subsection{Etiquette}
In short, the members of this class, both the instructor and the students, are expected to behave courteously and professionally in all interactions.  Under that umbrella statement, the following general guidelines should be followed in any class here at \ac{slu}.
	\begin{description}
	\item[Tolerance] Many of our discussions will be driven by opinions and based on challenging material.  Since we are all writers, everyone in class will have personal experiences and viewpoints that can contribute meaning to the conversations.  All participants are expected to treat others with dignity and respect and are expected to refrain from insensitive comments, including racist, ageist, sexist, classist, homophobic, or other disparaging and unwarranted views.
	\item[Timeliness] Students are expected to be ready for class at its designated time just as much as you expect the instructor to dismiss class by the designated time.  Should you arrive to class late for any reason, please do so with a minimum level of disruption.  If you need to leave class early for any reason, please notify the instructor in advance and be as non-disruptive as possible when leaving.
	\item[Phones] As a courtesy, all phones should be silenced during this or any other class. Should your phone accidentally create a distraction during class, you should take action to eliminate the distraction without adding to it.
	\item[Computers] You will need to use your computer in class regularly to collaborate with others and complete your assignments. Having the discipline of shutting off distractions (such as Facebook, chat applications, etc.) improves your ability to focus and participate meaningfully.
	\item[Messages] Grammar, spelling, and punctuation reflect the formality of the situation in which they appear.  Keep in mind that emails and discussion posts you write for this class are being read by an English teacher in a composition course.  Though I don't expect discussion posts to be perfectly error-free (they're not that important), I do expect you to treat written language with respect. Complete sentences and full words (``you'' instead of ``u'') are always a good idea, even if the intended audience is your peers.
	\item[Email] As \iac{slu} student, you have access to a student email account, which will be the primary method of communication for course-related announcements and information. Your instructor generally replies to messages within 24 hours Sunday through Thursday; messages sent on Fridays or Saturdays might get a delayed response.
	\end{description}
	

\subsection{Computer Reliability}\label{sub:reliability}
Save everything, and save often.  Computer problems are regular part of life, and I expect you to prepare for them rather than use them as an excuse for late work. Every semester, your instructor has students sustain a complete hard drive failure, losing all their work. Such failures are unavoidable, but losing data is not, if you plan ahead. Working backups and protection from Windows viruses are essential to avoid the most common catastrophes.  A free Dropbox account (\href{http://db.tt/mzWxY8s}{http://dropbox.com}) provides convenient and automatic backups, allows you to access your files from any networked computer in case disaster befalls yours, and preserves old versions of files so that if a file is deleted or altered, a previous copy can be restored. Regardless of the solution you choose, know how you will keep moving if your computer fails.

\subsection{Honor Code} % (fold)
\label{sub:honor_code}
Saint Leo University holds all students to the highest standards of honesty and personal integrity in every phase of their academic life. All students have a responsibility to uphold the Academic Honor Code by refraining from any form of academic misconduct, presenting only work that is genuinely their own, and reporting any observed instance of academic dishonesty to a faculty member.

Complete details can be found in the full \href{http://www.saintleo.edu/media/626793/academic_honor_code_policy.pdf}{\ac{slu} Academic Honor Code}, from which the above paragraph was excerpted. Additionally, \ac{slu}'s \href{http://www.saintleo.edu/about/florida-catholic-university.aspx}{Core Values} include Integrity, by which we ``pledge to be honest, just, and consistent in word and deed.''
% subsection honor_code (end)

\subsection{Commitment to Academic Excellence} % (fold)
\label{sub:commitment_to_academic_excellence}
Academic excellence is reflected by balance and growth in mind, body, and spirit that develops a more effective and creative culture for students, faculty, and staff. It promotes integrity, honesty, personal responsibility, fairness, and collaboration at all levels of the university. At the level of students, excellence means achieving mastery of the specific intellectual content, critical thinking, and practical skills that develop reflective, globally conscious, and informed citizens ready to meet the challenges of a complex world.
% subsection commitment_to_academic_excellence (end)

\subsection{Instructor's Research} % (fold)
\label{sub:instructor_s_research}
For the purposes of conducting research or improving his teaching practices, your instructor may use your work anonymously as an example in other classes, in workshops and lectures, or in publications. For example, I might quote from one of your assignments in a journal article or conference presentation, without revealing your identity. If you do \textbf{not} wish your work to be used in this manner, let me know in writing (via email is fine) within one week after the date your final grade is due. (This date is listed on \href{http://www.saintleo.edu/resources/academic-catalogs-schedules-calendars.aspx}{\ac{slu}'s Academic Calendar}.) Your course grade will not be affected by your decision to permit or deny my use of your work.\footnote{The ``Instructor's Research'' section is adapted from the syllabus of Beth Rapp-Young, \textsc{ucf}.}
% subsection instructor_s_research (end)


\section{Available Resources} % (fold)
\label{sec:available_resources}
\subsection{Library Resources} % (fold)
\label{sub:library_resources}
You may find that libraries and their resources, both online and on-ground, come in handy for this course. You have a few options, including but not limited to, the below:

\subsubsection{Daniel A.\ Cannon Memorial Library} % (fold)
\label{ssub:cannon_memorial_library}
Librarians are available in the University Campus library during reference hours to answer questions concerning research strategies, database searching, locating specific materials, and interlibrary loan (\textsc{ill}). Learn more about library services and check their hours by visiting their LibGuides page (\href{http://saintleo.libguides.com/calendar}{http://saintleo.libguides.com/calendar}) or search their catalog from their main page (\href{http://saintleo.edu/library}{http://saintleo.edu/library}).
% subsubsection cannon_memorial_library (end)

\subsubsection{Community Libraries} % (fold)
\label{ssub:community_libraries}
Almost all public library systems offer free borrowing privileges to local community members, as well as free access to their online databases, including access from your home.  The key is obtaining a library card.  Check with your local library to find out how to get a borrower’s card.  
% subsubsection community_libraries (end)

\subsubsection{The Library at \textsc{usf}} % (fold)
\label{ssub:the_library_at_}
University Campus students have borrowing privileges at the University of South Florida.  Be sure to bring a current Saint Leo student ID card and proof of current enrollment with you to borrow \textsc{usf} library books.
% subsubsection the_library_at_ (end)

% subsection library_resources (end)

\subsection{Writing Resources on Campus} % (fold)
\label{sub:writing_resources_on_campus}
While on University Campus, \ac{slu} students have access to two helpful resources targeted specifically at writing assistance. Basically, we offer two places where you can get free tutoring and after-class help with your writing.

\subsubsection{Writing and Research Instruction at the Library} % (fold)
\label{ssub:writing_and_research_instruction_at_the_library}
The Cannon Memorial Library now offers instruction in writing and research to students of all levels, across the curriculum. Ángel L.\ Jiménez and John David Harding offer instruction on all aspects and stages of the writing process. Please make an appointment by visiting their website (\href{http://saintleolibrary.cloudaccess.net/research-writing-help.html}{http://saintleolibrary.cloudaccess.net /research-writing-help.html}).

% subsubsection writing_and_research_instruction_at_the_library (end)

\subsubsection{Learning Resource Center} % (fold)
\label{ssub:learning_resource_center}
The \ac{lrc} provides tutoring services for all \ac{slu} students.  The \ac{lrc} is located on the second floor of the Student Activities Building and appointments are available through \href{http://tutoring.saintleo.edu/TracWeb40/Default.html}{TutorTrac} or on a walk-in basis. When attending a session you will need to bring: course syllabus, course notes and materials presented in class, course textbook(s), and any questions you have for the tutor.  An English tutor will be able to help you:

\begin{itemize}
	\item Understand assignment requirements
	\item Develop ideas
	\item Plan and organize your writing
	\item Identify and address some key aspects of your writing for you to revise
	\item Learn to cite and document sources
	\item Practice strategies for proofreading and editing
	\item Learn to correct errors in grammar, punctuation, and mechanics
\end{itemize}

% subsection writing_resources_on_campus (end)

\subsection{Accommodations} % (fold)
\label{sub:accommodations}
Students with disabilities who need accommodations in this course must contact the instructor at the beginning of the semester to discuss needed accommodations. No accommodations will be provided until the student has both contacted the Office of Disability Services [Student Activities Building 207, phone \mbox{(352)~588-8464}, fax \mbox{(352)~588-8605}, or email \href{mailto:adaoffice@saintleo.edu}{adaoffice@saintleo.edu}] and contacted the instructor to discuss appropriate accommodations.

More personally, I am dedicated to incorporating inclusive practices for all students within the classroom, as well as providing for specific accommodation requests. Beyond the provisions of the Office of Disability Services, please feel free to contact me with any suggestions and/or requests you have regarding the accessibility of information and/or interactions in this course. I am always interested in these types of suggestions, as they may not only meet a specific student's needs but could also be employed to make the overall class more accessible and inclusive for all students.\footnote{The second ¶ in the ``Accommodations'' section is adapted from the syllabus of Barbi Smyser-Fauble, \textsc{isu}.}

% subsection accommodations (end)

% subsection learning_resource_center (end)


% section available_resources (end)


\section{Works Cited}\label{works-con}
\renewcommand\refname~{\vspace{-22pt}}

\printbibliography
\end{document}


%\begin{landscape}
\clearpage%\ \newpage
%\addtocounter{page}{-2}
\section{Course Calendar}
\label{sec:calendar}
{\centering
\footnotesize %brace balanced at end of calendar
%\vspace{-1in}
\tablehead{\toprule\textbf{\textsc{Unit}} & \textbf{\textsc{Week}} & \textbf{\textsc{Date}} & \textbf{\textsc{Readings/Homework\newline(Before Class)}} & \textbf{\textsc{Guiding Question\newline(During Class)}}\\}
\tablelasthead{\toprule\textbf{\textsc{Unit}} & \textbf{\textsc{Week}} & \textbf{\textsc{Date}} & \textbf{\textsc{Readings/Homework\newline(Before Class)}} & \textbf{\textsc{Guiding Question\newline(During Class)}}\\}
\begin{mpxtabular}{>{\bfseries}p{0.75in}ccp{1.85in}p{1.85in}} % for portrait
%\begin{xtabular}{>{\bfseries}lccp{2.25in}p{3.25in}} % for landscape
%	\toprule\textbf{\textsc{Topic(s)}} & \textbf{\textsc{Week}} &\textbf{\textsc{Class Discussion}} & \textbf{\textsc{Readings/Homework}}\\

%%%%%%%%%%%%%%%%%%%%%%% Material below pasted from Numbers. Make edits there, not here.



\midrule	Reading 
in the University	&	1	&	6 Jan	&	n/a	&	What is a “Composition Class”?			\\
\cmidrule(l){3-5}		&		&	8 Jan	&	Swales, “Create a Research Space (CARS)” WAW 6–8
Syllabus Quiz	&	What are the tricks to reading a research article in an academic journal?	\newline\textbf{	\textsc{ucf} Drop/Swap Deadline Thursday	}\\
\midrule	The Writing Process	&		&	10 Jan	&	Perl, “Composing Process” WAW 191–215	&	How can we study the writing process if it’s all in our head?	\newline\textbf{	\textsc{ucf} Add Deadline	}\\
\cmidrule(l){2-5}		&	2	&	13 Jan	&	Skim EW 52–56; write self-portrait; record yourself writing it	&	What makes a person a “bad writer”?	\newline\textbf{	2-page Writer Self-Portrait Due	}\\
\cmidrule(l){3-5}		&		&	15 Jan	&	Lamott, “Shitty First Drafts” WAW 301–304
Keep a complete writing journal for two days.	&	What makes a document or essay “bad writing”?			\\
\cmidrule(l){3-5}		&		&	17 Jan	&	Berkenkotter and Murray, “Decisions and Revisions…” and “Response of a Lab Rat…” WAW 216–235
D/J 3--5	&	How can the audience influence what authors write?			\\
\cmidrule(l){2-5}		&	3	&	20 Jan	&\multicolumn{2}{c}{\textbf{	Martin Luther King, Jr. Day—No class		}}			\\
\cmidrule(l){3-5}		&		&	22 Jan	&	Transcribe your recording: How would you apply Perl’s process to it?	&	What can you look for in your transcriptions?			\\
\cmidrule(l){3-5}		&		&	24 Jan	&	Read Rose (WAW 236) or Williams (WAW 37)	&	What can you assert or conclude based on your observations?			\\
\cmidrule(l){2-5}		&	4	&	27 Jan	&	EW, 82–94 (Reviewing and Revising); review assignment sheet	&	What do we mean by “revising”? What are you qualified to review?			\\
\cmidrule(l){3-5}		&		&	29 Jan	&	Write shitty first draft of Process Analysis	&	What can other students do to make their papers as awesome as yours?			\\
\cmidrule(l){3-5}		&		&	31 Jan	&	Finalize Process Analysis	&	What have we figured out so far?	\newline\textbf{	Process Analysis Due	}\\
\midrule	Discourse Communities	&	5	&	3 Feb	&	Swales, “The Concept of Discourse Community” WAW 466–480	&	What is a Discourse Community?			\\
\cmidrule(l){3-5}		&		&	5 Feb	&	Review Characteristics assignment sheet; brainstorm groups	&	How do certain groups exhibit Swales’ characteristics?			\\
\cmidrule(l){3-5}		&		&	7 Feb	&	Brainstorm a list of new vocabulary words from this class	&	What are the effects of a community’s specific lexis?			\\
\midrule	\textmd{\emph{Lexia}}	&	6	&	10 Feb	&	Gee, “Literacy, Discourse, and Linguistics: Introduction” WAW 481–495	&	How can language control group membership?			\\
\cmidrule(l){3-5}		&		&	12 Feb	&	Choose a term with specific, rich meaning in an academic discourse; bring dictionary definition.	&	What does it mean to be “literate”?			\\
\cmidrule(l){3-5}		&		&	14 Feb	&	Find discipline-specific definition of your term.	&	What is “mushfaking”, and is it a good thing?			\\
\cmidrule(l){2-5}		&	7	&	17 Feb	&	Ask member of community how he/she learned the term	&	What are your non-dominant discourses?			\\
\cmidrule(l){3-5}		&		&	19 Feb	&	Write shitty first draft of Definition paper	&	How much can you learn about a group through the words they use?			\\
\cmidrule(l){3-5}		&		&	21 Feb	&	Identify traits/characteristics of academic articles	&	Why do we write the way(s) we do?	\newline\textbf{	Multi-Dimensional Definition Due	}\\
\midrule	\textmd{\emph{Genre}}	&	8	&	24 Feb	&	Devitt, “Generalizing about Genre” (Get from Webcourses), pp.\ 573–580	&	What “recurring situations” exist in academic writing?			\\
\cmidrule(l){3-5}		&		&	26 Feb	&	Write one message in two genres	&	What affordances and constraints are created by the genres we use?			\\
\cmidrule(l){3-5}		&		&	28 Feb	&	Porter, “Intertextuality and the Discourse Community” WAW 86–96	&	What is plagiarism?			\\
\cmidrule(l){2-5}		&	X	&	3--7 Mar	&\multicolumn{2}{c}{\textbf{	Spring Break—No class		}}			\\
\cmidrule(l){2-5}		&	9	&	10 Mar	&	Brainstorm groups that could benefit from an awareness of genre, intertextuality, or discourse communities	&	How do writers decide what genre to use for their ideas?			\\
\cmidrule(l){3-5}		&		&	12 Mar	&	Mirabelli, “Learning to Serve” WAW 538–555	&	How do genre and authority relate?			\\
\cmidrule(l){3-5}		&		&	14 Mar	&	Write shitty first draft of Genre Analysis	&	What can we learn about a situation by looking at the written responses to it?			\\
\cmidrule(l){2-5}		&	10	&	17--21 Mar	&\multicolumn{2}{c}{\textbf{	Friend at Conference—No class (\textsc{ucf} Withdrawal Deadline 3/18)		}}			\\
\midrule	\textmd{\emph{Authority}}	&	11	&	24 Mar	&	Wardle, “Identity, Authority, and Learning to Write in New Workplaces” WAW 520–537
D/J 4, 6	&	What could we have told Alan to make things work?	\newline\textbf{	Genre Analysis Due	}\\
\cmidrule(l){3-5}		&		&	26 Mar	&	Read Science Accommodation sheet	&	What kind of authority can you bring to your papers? to your major?			\\
\cmidrule(l){3-5}		&		&	28 Mar	&	Penrose and Geisler, “Reading and Writing Without Authority” WAW 602–617
D/J 1, 2	&	What advice do you have for Janet? for Roger?			\\
\cmidrule(l){2-5}		&	12	&	31 Mar	&	McCarthy, “Stranger in Strange Lands” WAW 667–699
D/J  1, 3, 7	&	What advice do you give to Dave? What advice does McCarthy give you?			\\
\cmidrule(l){3-5}		&		&	2 Apr	&	Bring in article on science finding	&	Which of your two articles is better? (Why is that a trick question?)			\\
\cmidrule(l){3-5}		&		&	4 Apr	&	Keller, “Studies Explore…” WAW 595–601
D/J 1, 2	&	How different can two articles be if they’re about the same thing?			\\
\cmidrule(l){2-5}		&	13	&	7 Apr	&	Write shitty first draft of Science Accommodation	&	How can your writing show the conversation?			\\
\cmidrule(l){3-5}		&		&	9 Apr	&	Grant-Davie, “Rhetorical Situations and Their Constituents” in WAW textbook
D/J 1, 3, 5	&	How can writing be a negotiation?			\\
\midrule	Rhetorical\newline Situations	&		&	11 Apr	&	Haas and Flower, “Rhetorical Reading Strategies and the Construction of Meaning” WAW 120–38
D/J 1, 3	&	Who are Haas and Flower writing to, and how can you tell? How does a reader construct new meaning while reading?	\newline\textbf{	Science Accommodation Due	}\\
\cmidrule(l){2-5}		&	14	&	14 Apr	&	Bring in informational textual artifact. Brainstorm controversial topics.	&	What are the rhetorical situations and exigencies of “informational” texts? What are the constraints of writing for school?			\\
\cmidrule(l){3-5}		&		&	16 Apr	&	Write shitty first draft of Navigating Sources	&	What are you good seeing that can make other students’ papers better?			\\
\cmidrule(l){3-5}		&		&	18 Apr	&	Finalize draft	&	What’s left to do?	\newline\textbf{	Navigating Sources Due	}\\
\midrule	Final\newline Portfolios	&	15	&	21 Apr	&	EW 99–101; write Course Audit cover letter; revise major papers; prepare final portfolio	&	How do your writing and revisions demonstrate that you “got” the course? Are you a Stranger in Strange Lands?			\\
\cmidrule(l){2-5}		&	16	&	Exams	&	Finalize your portfolio	&	Any last-minute panic attacks?	\newline\textbf{	Portfolio Due	}\\





%%%%%%%%%%%%%%%%%%%%%%% Material above pasted from Numbers. Make edits there, not here.
 
	\bottomrule
    \end{mpxtabular}
} %matches the \footnotesize at top of calendar
%    \end{center}

\subsection{Changes}
    Material in the preceding schedule is subject to change at the discretion of the instructor.  Students will be notified of any changes in class.  If relevant, changes will also be reflected on \href{https://webcourses.ucf.edu/webct/logon/15263651955041}{Webcourses}.

\subsection{Final Exams} % (fold)
\label{sub:final_exams}
Because this class includes a portfolio that documents your progress over the semester, there is no final exam. However, all instructors at \textsc{ucf} are required to hold class during exam periods. Therefore, for students having trouble submitting their final portfolios, a troubleshooting class will be held during the exam periods below. These class sessions will meet in the Texts \& Technology Lab, located in \textsc{cnh 207c},  
	\ifsecondclass
		\textbf{Wednesday, 23 April 2014, 10:00--12:50}
	\else % 1030 only
		\textbf{Monday, 28 April 2014, 10:00--12:50}
	\fi.
% subsection final_exams (end)
  
%    \end{landscape}


\section{Works Cited}\label{works-con}
\renewcommand\refname~{\vspace{-22pt}}

\printbibliography


\end{document}  

